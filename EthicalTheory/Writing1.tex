\documentclass[12pt]{article}
\usepackage[left=3cm, right=3cm, top=2cm]{geometry} 
\setlength{\parindent}{4em}
\setlength{\parskip}{1em}
\renewcommand{\baselinestretch}{1.5}
\begin{document}
\title{Utilitarian Viewpoints}
\author{Dylan Orris}

\maketitle

Two views of the application of utilitarianism exist, the total view and the prior existence. Prior existence, as the name implies, refers to valuing the preference or pleasure of a being already in existence. The total view values both existing and hypothetical beings. For example, the total existence view sees no difference in the death of a person, should another person be created as a result of the destruction of the original. The prior existence view however would take issue with the destruction of life, regardless of the possible happiness of the being which would result from the death. These views need not contradict each other, as each has a domain where its application is most fit. The total utilitarian view should be applied to non-persons such as most animals, while the prior existence view should be applied to persons.

First, let us define a `non-person'. This is not meant to mean non-human animals, rather beings which are sentient, but not conscious. An example of this would be a caterpillar. Though this animal is clearly alive, it has no desire for the future and no conception of the self. For this reason, sentences such as `The caterpillar feels sad' are incorrect. To say the caterpillar feels anything beyond physical stimulus is a misattribution. Persons on the other hand \textit{do} have desires for the future, and an understanding of themselves as beings.

Why does this mean different views should be applied? Caterpillars and other non-persons lack consciousness, and thus lose nothing when they die. The world loses a source of potential happiness, but should that death increase the happiness of others an equivalent amount, no real change has occurred. Just as an identical copy of the caterpillar being created by the death of the first would have no preference to come into being (both due to its lack of possible consciousness and its lack of existence) the existing caterpillar also has no preferences. As no violation of preferences occurs, and total happiness remains the same, the application of the total view of utilitarianism is most justified in this situation.

Persons, however, understand the concept of `I', and thus have a preference to avoid death. When we consider a person and their identical clone, each clone would say of the other `That is not me!'. If that clone required that the original be executed to come into existence, regardless of if the clone would have the same potential for happiness as the original, it would be immoral to create the clone. Prior to existing, the clone has no preferences. It cannot be disappointed by never being created, whereas the living person can be caused great distress by being executed. As the being in existence is harmed in this scenario, while the future being is not, the decision to apply the prior existence view in this case is obvious.

The difference in application of the total and prior existence views comes down to a question of consciousness. Non-persons, those which have no consciousness, do not experience in the same way as a conscious being. To replace them with an identical copy is acceptable, as no preferences are violated. Persons understand `I', which allows for a preference of the continued existence of the self - this does not oppose desires of a potential copy, as the copy does not exist so has no desire. By understanding this, the decision of when to apply the total view or the prior existence view is clear.
\end{document}
